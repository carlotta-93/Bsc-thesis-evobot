\chapter{Il sistema chimico}
\vspace{0.5cm}
\label{cha:789}

Se consideriamo la cellula, in termini generali, come un sistema chimico aperto in una situazione di non-equilibrio, è essenziale avere a disposizione un rifornimento di materiale fresco e di energia per sostenere questo sistema. Per far ciò che questo avvenga, la cellula modifica il suo ambiente esterno metabolizzando risorse di sostentamento e producendo dei prodotti e degli scarti. Per evitare una situazione statica, di equilibrio, il sistema deve trovare in qualche modo nuove risorse da sfruttare e allo stesso tempo deve evitare gli eventuali effetti inibitori dei prodotti di scarto. In questo senso puramente chimico-biologico si crede che l'abilità di movimento giochi un ruolo importante per evitare lo stato di equilibrio nella creazione di sistemi cellulari artificiali. \cite{doi:10.1021/ja0706955}

Il nuovo sistema chimico su cui si basa lo studio è composto da \emph{droplets} di Decanolo che si muovo in una soluzione acquosa di Acido decanoico lungo gradienti di concentrazione di Cloruro di Sodio
.

\section{Chemiotassi e Chemochinesi}
\label{sec:456}
movimento secondo gradiente 


    

\subsection{L'importanza del movimento}
\label{sec:00456}
\begin {itemize}
	\item nutrimento di cellule singole e sopravvivenza + 
	\item condizioni di malattia e di salute
	\item trasporto di cargo
\end{itemize}

\section{Componenti inorganiche}
\label{sec:123}
decanolo
decanoato
NaCl  molarità  pH 


