\chapter*{Sommario} % senza numerazione
\label{sommario}

\addcontentsline{toc}{chapter}{Sommario} % da aggiungere comunque all'indice

“The Three Laws of Robotics:

\noindent 1: A robot may not injure a human being or, through inaction, allow a human being to come to harm;

\noindent 2: A robot must obey the orders given it by human beings except where such orders would conflict with the First Law;

\noindent 3: A robot must protect its own existence as long as such protection does not conflict with the First or Second Law;

\noindent The Zeroth Law: A robot may not harm humanity, or, by inaction, allow humanity to come to harm.”


Sommario è un breve riassunto del lavoro svolto dove si descrive l'obiettivo, l'oggetto della tesi, le metodologie e le tecniche usate, i dati elaborati e la spiegazione delle conclusioni alle quali siete arrivati.  

Il sommario dell’elaborato consiste al massimo di 3 pagine e deve contenere le seguenti informazioni:
\begin{itemize}
  \item contesto e motivazioni 
  \item breve riassunto del problema affrontato
  \item tecniche utilizzate e/o sviluppate
  \item risultati raggiunti, sottolineando il contributo personale del laureando/a
\end{itemize}