\chapter{Conclusioni e Sviluppi Futuri}
\vspace{0.5cm}
\label{cha:789}

\section{Problemi nella struttura del robot}
\label{sec:456}
La struttura di EvoBot è stata progettata seguendo il modello di una stampante 3D. I layers che lo costituiscono sono ancorati allo scheletro esterno con delle leve bloccanti. Questo fa pensare che la struttura sia facilmente regolabile dall'utente. In realtà spostare i layers risulta alquanto complesso e macchinoso se non con l'aiuto di un'altra persona. L'\emph{actuation layer} contiene la testa del robot e sia i cavi che collegano la scheda Arduino che il cavo di corrente dei motori che muovono la testa del robot sono esterni alla cornice in metallo per cui durante il movimento bisogna prestare attenzione a posizionarli nel modo corretto.

\subsection{La testa ed il modulo siringa}
La \emph{head} di EvoBot è composta da due PCBs (printed circuit board), circuiti stampati costituiti da connettori a molla per creare i contatti elettrici con i moduli. L'inserimento e l'estrazione del modulo siringa deve avvenire quindi con delicatezza in quanto ad ogni movimento i connettori possono danneggiarsi ed il modulo resterebbe elettricamente scollegato.

Tutte le componenti del modulo siringa sono state prodotte utilizzando una stampante 3D. Il materiale di produzione risulta quindi in plastica rigida. Questa proprietà conferisce ad ogni pezzo rigidità ma allo stesso tempo fragilità. Durante le prime fasi del lavoro il bastoncello a spirale, che unisce lo stantuffo al suo motore per il movimento, tendeva a perdere il passo all'interno del motore causando la fuoriuscita di questo dalla base in cui era fissato. Questo imprevisto ci ha impedito di proseguire con gli studi svariate volte in quanto non vi era a disposizione un elevato numero di moduli siringa. Il problema è stato risolto nelle siringhe di nuova generazione fissando un blocco con delle viti sulla parte superiore della base che contiene il motore. 
Un altro problema incontrato risulta per il lavaggio del corpo delle siringhe: queste sono fissate con un sostegno e delle viti. E' necessario svitare il sostegno ed estrarre la siringa per eseguire almeno un lavaggio del corpo della siringa con etanolo così che che non restino residui di nessuna soluzione all'interno. Il lavaggio dovrebbe avvenire alla fine di ogni giornata di lavoro; svitare ogni volta il sostegno, a lungo termine, potrebbe rovinare il filetto del buco in cui è avvitata la vite. Per evitare questo passaggio si è fatto in modo di gestire quantità di liquido sufficientemente piccole da restare nel puntale della siringa e non entrare nel corpo. Per gli esperimento condotti si è riusciti a rispettare il limite, tuttavia in fututo potrebbe esserci la necessità di utilizzare dei volumi maggiori. Allora si potrebbe risolvere il problema implementando un ciclo di "lavaggio" della siringa tirando su e spingendo fuori dell'etanolo da una Petri posizionata sull'\emph{experimental layer}. 

\section{Conduzione degli esperimenti}
\label{sec:123}
La struttura del robot prevedeva una lastra di plastica bianca da apporre sopra il modulo siringhe per avere la visuale di una superficie più omogenea nel video registrato dalla camera. Lo strato di plastica non era presente al momento dell'arrivo del robot nel laboratorio quindi si è ricorsi a sostituirlo con un 
La prima difficoltà incontrata nella conduzione degli esperimenti era la difficoltà

\subsection{influenza agenti esterni}
\subsection{sistemare i parametri e automazione}