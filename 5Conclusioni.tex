\chapter{Conclusioni e Sviluppi Futuri}
\vspace{0.5cm}
\label{cha:789}

\section{Argomentazione dei risultati}

\section{Problemi nella struttura del robot}
\label{sec:456}
La struttura di EvoBot è stata progettata seguendo il modello di una stampante 3D. I layers che lo costituiscono sono ancorati allo scheletro esterno con delle leve bloccanti. Questo fa pensare che la struttura sia facilmente regolabile dal singolo utente. In realtà spostare i layers risulta alquanto complesso e macchinoso se non con l'aiuto di un'altra persona. L'\emph{actuation layer} contiene la testa del robot e sia i cavi che collegano la scheda Arduino che il cavo di corrente dei motori che muovono la testa del robot sono esterni alla cornice in metallo per cui durante il movimento bisogna prestare attenzione a posizionarli nel modo corretto.
\\La struttura del robot prevede la presenza di un piano di plastica bianca da posizionare al di sopra del modulo siringhe per avere una superficie più omogenea nello sfondo del video registrato dalla camera. Lo strato di plastica non era presente al momento dell'arrivo del robot nel laboratorio quindi si è ricorsi a sostituirlo con un cartone di colore bianco. Questo ha permesso un migliore riconoscimento della forma della droplet avendo come sfondo una superficie omogenea. 

\subsection{La testa ed il modulo siringa}
La \emph{head} di EvoBot è composta da due PCBs (printed circuit board), circuiti stampati costituiti da connettori a molla per creare i contatti elettrici con i moduli. L'inserimento e l'estrazione del modulo siringa deve avvenire quindi con delicatezza in quanto ad ogni movimento i connettori possono danneggiarsi ed il modulo resterebbe elettricamente scollegato.
\\Tutte le componenti del modulo siringa sono state prodotte utilizzando una stampante 3D. Il materiale di produzione risulta quindi in plastica rigida. Questa proprietà conferisce ad ogni pezzo rigidità ma allo stesso tempo fragilità.
\\Un esempio di questa fragilità è stata la rottura della base contenente il motore dello stantuffo. Durante le prime fasi del lavoro il bastoncello a spirale, che unisce lo stantuffo al suo motore, per il movimento verticale, tendeva a perdere il passo all'interno del motore causando la fuoriuscita di questo dalla base in cui era fissato. 
Questo imprevisto ha comportato numerose interruzioni delle attività in quanto non si disponeva dei moduli siringa supplementari necessari per la sostituzione. Il problema è stato risolto nella struttura delle siringhe di nuova generazione fissando un blocco con delle viti sulla parte superiore della base che contiene il motore. 
Un altro dei problemi incontrati ha riguardato il lavaggio del corpo delle siringhe in quanto esse sono alloggiate in dei sostegni ad incastro. E’ pertanto necessario estrarla per eseguire almeno un lavaggio del corpo della siringa con etanolo così che non restino residui di nessuna soluzione all'interno. Il lavaggio dovrebbe avvenire alla fine di ogni giornata di lavoro ed estrarre la siringa dal sostegno, a lungo termine potrebbe portare alla rottura di questo. Il rimedio adottato è stato quello di utilizzare quantità di liquido sufficiente ridotte in modo che il loro volume restasse contenuto nel puntale senza invadere il corpo della siringa, almeno per quanto riguarda il decanolo, sostanza tossica che può provocare irritazioni oculari. Per gli esperimenti condotti si è riusciti a rispettare il limite ma in futuro si potrebbe aver bisogno di quantità di liquidi maggiori. Si è pensato ad una possibile soluzione: implementare un ciclo di "lavaggi" della siringa aspirando e spingendo fuori ripetutamente dell'etanolo da un pozzetto posizionato sull'\emph{experimental layer}. L'etanolo è un alcol incolore impiegato come solvente in laboratorio; è adatto a solubilizzare sia sostanze organiche che inorganiche. 
Nel capitolo 4 si è parlato dell'influenza della punta della siringa nel momento in cui vengono inseriti i liquidi nel sistema. Il fattore da tenere quindi in considerazione è la velocità con cui il liquido viene immesso, ovvero la velocità di spostamento del plunger. Per controllare questa specifica è stata implementata una API specifica da lanciare dopo la calibrazione della testa del robot. Diversamente dalla calibrazione però, l'impostazione non viene mantenuta dopo lo spegnimento del robot. Questo porta a dover rilanciare il programma ogni volta che si inizia un nuovo esperimento. Ci si propone quindi di trovare una soluzione da implementare direttamente nel programma di raccolta dati così da ottimizzare il tempo di esecuzione degli esperimenti.  

\section{Conduzione degli esperimenti}
\label{sec:123}
Come accennato nel capitolo 4 di questo studio, è possibile che alcuni degli esperimenti condotti siano stati in parte influenzati da fattori esterni alla struttura del robot. Inoltre ci si propone, per studi futuri, di automatizzare la gestione dell'esperimento.
 
\subsection{Influenza di agenti esterni} 
La cornice che racchiude i \emph{layers} non ha una chiusura ermetica per cui i flussi di aria creati attorno al robot possono aver influito sul movimento della droplet. Questo problema è stato risolto in parte apponendo dei fogli di carta sui lati esterni ma risulta evidente la necessità di una copertura più robusta. Un flusso di aria può influire sul sistema facendo muovere la droplet più velocemente o spostandola in zone in cui il gradiente di sale non è sufficientemente forte da attirarla. I valori del percorso fatto e del tempo impiegato non risulterebbero quindi estremamente puntuali. 
Questo problema ci ha condotto a registrare il punto di inizio della droplet non come quello in cui questa viene inserita dalla siringa ma come quello in cui viene riconosciuta dal programma.  
Infatti le posizioni C e D mostrate nella figura 4.2, sono arbitrarie e variano leggermente da esperimento ad esperimento. Si è cercato tuttavia, per mantenere lo spazio degli esperimenti sufficientemente uniforme, di seguire sempre lo schema proposto dalla figura 4.2. Questo non è sempre stato possibile in quanto appunto, la droplet posizionata in posizione C, essendo influenzata dai moti convettivi, si muoveva in altre posizioni prima che il sale venisse inserito nel sistema. 

\subsection{Modifiche nel protocollo e automazione}
Gli sviluppi futuri di questo studio prevedono l'apporto di modifiche al protocollo usato e l'automazione dei vari processi che lo compongono.
Un primo parametro da modificare è la distanza tra il sale e la droplet, questa potrebbe assumere un valore definito in precedenza. Tuttavia, per poter avere la certezza che la distanza abbia sempre lo stesso valore si devono prima risolvere i problemi relativi allo spostamento involontario della droplet. Per facilitare questa specifica si potrebbe pensare di eseguire gli esperimenti su una Petri rettangolare e non rotonda, considerando però che questo ci distanzierebbe dall'ambiente reale in cui si muovono le cellule organiche, più vasto e meno limitato. 
Un altro parametro che si può decidere di impostare è quello relativo al tempo della modalità di tracciamento. Se infatti si impostasse un tempo uguale per tutte le combinazioni si potrebbero osservare anche altre caratteristiche del movimento. 
Entrambe queste modifiche verrebbero applicate creando un programma per controllare il robot in modo automatico. Si possono impostare automaticamente i movimenti delle siringhe, della testa del robot e le quantità di liquido da aspirare ed immettere nel sistema. Si avrebbero così degli esperimenti identici in ogni ripetizione, più veloci nell'esecuzione e ripetibili un numero maggiore di volte.

\subsection{Conslusioni generali}

aggiungi anche un pezzettino in cui dici che il 5mM è pià fico del 20, e che dagli esperimenti che hai fatto puoi dire la droplet si muove meglio in certe condizioni (e specifichi quali) e peggio in altre e specifichi, come già elencato nei risultati… nn è una ripetiizione, in questa parte dovresti argomentare i tuoi risultati nn solo specificare tutta la roba che nn ha fuznionato, discuti ptima i rusltati e poi aggiungi tutte le critiche dopo aver messo una frase iniziale nella quale specifichi le critihce servono come base per formular ele prospettive future
