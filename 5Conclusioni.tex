\chapter{Conclusioni e Sviluppi Futuri}
\vspace{0.5cm}
\label{cha:789}
\section{Problemi nella struttura del robot}
\label{sec:456}
La struttura di EvoBot è stata progettata seguendo il modello di una stampante 3D. I layers che lo costituiscono sono ancorati allo scheletro esterno con delle leve bloccanti. Questo fa pensare che la struttura sia facilmente regolabile dal singolo utente. In realtà spostare i layers risulta alquanto complesso e macchinoso se non con l'aiuto di un'altra persona. L'\emph{actuation layer} contiene la testa del robot e sia i cavi che collegano la scheda Arduino che il cavo di corrente dei motori che muovono la testa del robot sono esterni alla cornice in metallo per cui durante il movimento bisogna prestare attenzione a posizionarli nel modo corretto.
\\La struttura del robot prevedeva la presenza di un piano di plastica bianca da posizionare al di sopra del modulo siringhe per avere una superficie più omogenea nello sfondo del video registrato dalla camera. Lo strato di plastica non era presente al momento dell'arrivo del robot nel laboratorio quindi si è ricorsi a sostituirlo con un cartone di colore bianco. Questo ha permesso un migliore riconoscimento della forma della droplet avendo come sfondo una superficie omogenea. 

\subsection{La testa ed il modulo siringa}
La \emph{head} di EvoBot è composta da due PCBs (printed circuit board), circuiti stampati costituiti da connettori a molla per creare i contatti elettrici con i moduli. L'inserimento e l'estrazione del modulo siringa deve avvenire quindi con delicatezza in quanto ad ogni movimento i connettori possono danneggiarsi ed il modulo resterebbe elettricamente scollegato.
\\Tutte le componenti del modulo siringa sono state prodotte utilizzando una stampante 3D. Il materiale di produzione risulta quindi in plastica rigida. Questa proprietà conferisce ad ogni pezzo rigidità ma allo stesso tempo fragilità.
\\Durante le prime fasi del lavoro il bastoncello a spirale, che unisce lo stantuffo al suo motore, per il movimento verticale, tendeva a perdere il passo all'interno del motore causando la fuoriuscita di questo dalla base in cui era fissato. Questo imprevisto ci ha impedito di proseguire con gli studi svariate volte in quanto non vi era a disposizione un elevato numero di moduli siringa da poter sostituire. Il problema è stato risolto nelle siringhe di nuova generazione fissando un blocco con delle viti sulla parte superiore della base che contiene il motore. 
Un altro problema incontrato risulta per il lavaggio del corpo delle siringhe: queste sono fissate con un sostegno e delle viti. E' necessario svitare il sostegno ed estrarre la siringa per eseguire almeno un lavaggio del corpo della siringa con etanolo così che che non restino residui di nessuna soluzione all'interno. Il lavaggio dovrebbe avvenire alla fine di ogni giornata di lavoro; svitare ogni volta il sostegno, a lungo termine, potrebbe rovinare il filetto del buco in cui è avvitata la vite. Per evitare questo passaggio si è fatto in modo di gestire quantità di liquido sufficientemente piccole da restare nel puntale e non entrare nel corpo della siringa. Per gli esperimenti condotti si è riusciti a rispettare il limite ma in futuro si potrebbe aver bisogno di quantità di liquidi maggiori. Si è pensato ad una possibile soluzione: implementare un ciclo di "lavaggi" della siringa tirando su e spingendo fuori dell'etanolo da una Petri posizionata sull'\emph{experimental layer} ripetute volte. 


\section{Conduzione degli esperimenti}
\label{sec:123}

\subsection{influenza agenti esterni}
Come accennato nel capitolo 4 di questo studio, è possibile che alcuni degli esperimenti condotti siano stati in parte influenzati da fattori esterni alla struttura del robot. La cornice che racchiude i \emph{layers} non ha una chiusura ermetica per cui i flussi di aria creati al robot possono aver influito sul movimento della droplet. Questo problema è stato risolto in parte apponendo dei fogli di carta sui lati esterni ma risulta evidente la necessità di una copertura più robusta. Un flusso di aria può influire sul sistema facendo muovere la droplet più velocemente o spostandola in zone in cui il gradiente di sale non è sufficientemente forte da attirarla. I valori del percorso fatto e del tempo impiegato non risultano quindi realistici.

 
 

\subsection{sistemare i parametri e automazione}

Le posizioni C e D all'interno della Petri sono arbitrarie e variano leggermente di esperimento in esperimento. Si è cercato tuttavia, per mantenere lo spazio degli esperimenti sufficientemente uniforme, di seguire sempre lo schema proposto in figura. Questo non è sempre stato possibile in quanto la droplet posizionata in posizione C era influenzata da moti convettivi creati dal movimento di aria attorno alla struttura del robot e dall'inserimento del puntale della siringa sulla superficie del decanoato presente nella Petri dish. 