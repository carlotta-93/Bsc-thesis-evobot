\chapter{La piattaforma robot}
\vspace{0.5cm}

\label{cha:789}
EvoBot unisce elementi dell'open source delle stampanti 3D e della robotica modulare per fornire a chimici, microbiologi e ricercatori nel campo della vita, a livello chimico, artificiale uno strumento economico ed estendibile basato su una piattaforma robotica open source per il trattamento di liquidi.
\\ La particolarità di Evobot sta nella progettazione mirata ad una interazione continua e in real time dell'utente con l'esperimento in corso. \cite{introd-robot}
La struttura di EvoBot ricorda quella di una comune stampante 3D, organizzata in strati.


\section{La componente hardware}
\label{sec:456}
introduzione - 3d printer like
\begin{itemize}
  \item arduino - ramps
  \item layers 
  \item head
  \item siringhe
\end{itemize}


\subsection{sottocapitolo}
\label{sec:00456}
come si fa un sottocapitolo


\section{La componente software}
\label{sec:123}
scrivere qualcosa di introduzione
\begin{itemize}
  \item API 
  \item bitbucket
  \item calibrazione
  \item processi visuali - threshold 
\end{itemize}


