\chapter{Final Multi Tracker}
\vspace{0.5cm}
\label{cha:789}

dire qualcosa


\section{Image processing}
\label{sec:456}
Si utilizza un set predefinito di colori per aiutare l'utente a scegliere la \emph{droplet} desiderata. In caso di colori non convenzionali, dovuti a condizioni di luce senza limitazioni e con svariate variazioni, l'utente può trovare ed impostare i parametri HSV, Hue Saturation Value (tonalità, saturazione e valore), sulla schermata di dialogo con la camera. L'utente ha anche la possibilità di interagire con EvoBot muovendone la testa, il corpo delle siringhe e gli stantuffi.
Questa funzionalità permette di tracciare ogni tipo di colore. Per fare in modo che la tracciabilità sia più stabile possibile, si può impostare un'area specifica, centrando per esempio il centro della Petri dish e regolando il diametro così da coprirne l'intera superficie. In questo modo l'ambiente circostante viene escluso dall'applicazione dei filtri per il riconoscimento dei colori da ricercare. 
Negli esperimenti condotti si è utilizzata una singola  \emph{droplet} di colore rosso. 




\subsection{droplet detection}
\label{sec:00456}
\begin {itemize}
	\item nutrimento di cellule singole e sopravvivenza + 
	\item condizioni di malattia e di salute
	\item trasporto di cargo
\end{itemize}

\section{csv}
\label{sec:123}



