\chapter{Raccolta dati ed analisi}
\vspace{0.5cm}
\label{cha:789}

Tra i parametri sperimentati si è cercata la combinazione "migliore": quella che permette alla droplet di muoversi nel minor tempo e di arrivare il più possibile vicino al punto in cui il sale è stato inserito nel sistema. Questo è stato possibile analizzando i diversi comportamenti della droplet in ognuna delle nove combinazioni. La scelta è stata fatta tenendo in considerazione che l'obiettivo dello studio è quello di fare in modo che la droplet si muova velocemente lungo un percorso specifico e definito. Lo studio di questo movimento di chemiotassi ci aiuta a comprendere e riprodurre le dinamiche di movimento delle cellule viventi. 

\section{Il software}
Il software implementato per la raccolta dei dati è stato creato utilizzando le API della piattaforma robotica per gestire il robot e sfruttando delle funzioni avanzate della libreria OpenCV.  



\section{Riconoscimento della droplet}
Si utilizza un set predefinito di colori per aiutare l'utente a scegliere la \emph{droplet} desiderata. Le droplets infatti possono essere colorate non solo con il rosso ma anche con colori scelti dall'utente in base alle necessità di ogni esperimento. 
Il software è implementato in modo che 
\\In caso di colori non convenzionali ed in presenza di condizioni di luce non ottimali, l'utente può trovare ed impostare i parametri HSV, Hue Saturation Value (tonalità, saturazione e valore), sulla schermata di dialogo con la camera. Questa funzionalità permette di tracciare ogni tipo di colore. 



\section{Image processing}
\label{sec:456}


L'utente ha anche la possibilità di interagire con EvoBot muovendone la testa, il corpo delle siringhe e gli stantuffi.
Per fare in modo che la tracciabilità sia più stabile possibile, si può impostare un'area specifica, centrando per esempio il centro della Petri dish e regolando il diametro così da coprirne l'intera superficie. In questo modo l'ambiente circostante viene escluso dall'applicazione dei filtri per il riconoscimento dei colori da ricercare. 
Negli esperimenti condotti si è utilizzata una singola  \emph{droplet} di colore rosso. 




\subsection{}
\label{sec:00456}

\section{csv}
\label{sec:123}



