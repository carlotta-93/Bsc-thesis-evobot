
\chapter{Introduzione}
\vspace{0.5cm}

\label{cha:intro}
Nel 1912 appare per la prima volta il termine "biologia sintetica" coniato dal biologo francese Stephane Leduc nel libro "La Biologie Syntetique". 
Il termine viene successivamente ripreso nel 1974 dal genetista polacco Wacław Szybalski:
\\ «Discutiamo ora del seguente problema, ovvero cosa avverrà dopo? Fino ad ora abbiamo lavorato sulla fase descrittiva della biologia molecolare. Ma la vera sfida partirà quando entreremo nella fase della sintesi biologica[...]. Io non credo che esauriremo idee nuove ed eccitanti[...] nella biologia sintetica.»
(Waclaw Szybalski) \cite{waclaw}


\section{Il progetto Evobliss}
\label{sec:context}
chi partecipa, come nasce, cosa fanno in generale.

\cite{ictbusiness}
\cite{donoho}


\section{Cosa si fa a Trento}
\label{sec:trento}
cosa facciamo noi ?


\section{La biologia artificiale}
\label{sec:artificial}
forse

