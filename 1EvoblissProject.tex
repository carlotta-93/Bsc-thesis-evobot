\chapter{Introduzione}
\vspace{0.5cm}

\label{cha:intro}
\epigraph{“Why is it less acceptable to seek how to make a cell than how to make a molecule?"}{--- \textup{Stéphane Leduc}}

"Come è iniziata la vita?": la domanda più interessante che l'umanità si sia mai posta. La necessità di trovare una spiegazione al fenomeno dell'origine della vita ha portato il mondo della ricerca a soffermarsi sulla possibile riproducibilità in laboratorio di aspetti semplici degli organismi viventi. 
Nel 1911 Stéphane Leduc, biologo francese, scrisse: «[...] la sintesi della vita deve avvenire nella produzione di forme intermedie tra il mondo organico ed il mondo inorganico, forme che possiedono soltanto alcuni rudimentali attributi di vita, alle quali altri attributi verranno aggiunti nel corso dello sviluppo dall'azione dell'ambiente circostante». Questa unione di elementi provenienti dal mondo vivente e dal mondo "non vivente" ci permettere di iniziare la costruzione di strutture semplici per poi svilupparle in forme sempre più simili ad "esseri viventi". 
Lo studio affrontato in questo progetto si avvale delle tecniche della la biologia sintetica per tentare di riprodurre comportamenti molto semplici, il movimento nel nostro caso, di ciò che potrà diventare un "essere vivente" sintetizzato in laboratorio.

\section{La Biologia Sintetica}
\label{sec:artificial}
Nel 1912 appare per la prima volta il termine "biologia sintetica" coniato dal biologo francese Stéphane Leduc (1853-1939) nel libro "La Biologie Syntetique". Egli affermò: «Quando avremo modo di conoscere il meccanismo fisico della produzione di un oggetto o di un fenomeno,[...] diventa possibile[...] riprodurre l'oggetto o il fenomeno, in quel momento la scienza è diventata sintetica. La biologia è una scienza come le altre, [...] deve essere in successione: descrittiva, analitica e sintetica». 
Lo scopo del nostro progetto è  quello di riprodurre caratteristiche base degli organismi viventi, nello specifico ci si è soffermati sul movimento in un sistema chimico.
\\Il termine viene successivamente ripreso nel 1974 dal genetista polacco Wacław Szybalski: «Discutiamo ora del seguente problema, ovvero cosa avverrà dopo? Fino ad ora abbiamo lavorato sulla fase descrittiva della biologia molecolare. Ma la vera sfida partirà quando entreremo nella fase della sintesi biologica[...]. Io non credo che esauriremo idee nuove ed eccitanti[...] nella biologia sintetica.» \cite{waclaw} 
\\Al fine di "sintetizzare" un essere vivente, le caratteristiche da ricercare sono:
\begin{itemize}
\item un corpo: per distingue il soggetto dall'ambiente circostante
\item un metabolismo: processo per il quale il soggetto prende risorse dall'ambiente e le trasforma in sostanze di mantenimento
\item informazione ereditabile: contenuta nel genoma ed in grado di essere passata alle generazioni successive
\end{itemize}
L'unione dei primi due punti ci porta ad avere un organismo in grado di muoversi e replicarsi, se in seguito aggiungessimo la capacità di passare l'informazione genomica avremmo così realizzato un soggetto evolutivo.
La Biologia Sintetica prevede due approcci diversi, nel primo caso si cerca di ricreare una nuova funzione e applicazione riprogrammando organismi già esistenti. Il secondo approccio prevede la sintesi da zero di forme di vita artificiale con alcune funzionalità degli esseri viventi aggiungendo singoli componenti al sistema. Il progetto si è focalizzato su questo approccio. Varie sostanze chimiche vengono utilizzate e riorganizzate in strutture e reti che si comportano come sistemi viventi. I sistemi vengono alimentati con opportuni nutrienti e viene osservato il loro comportamento, nello specifico il movimento, altra caratteristica fondamentale di un organismo vivente.
\pagebreak
\section{Il progetto EVOBLISS}
\label{sec:context}
Il progetto EVOBLISS  è un progetto di ricerca europeo finanziato dalla Commissione Europea FET (Future and Emerging Technologies). A questo progetto partecipano i laboratori di cinque diverse università europee:
\begin{itemize}
\item IT University of Copenhagen (ITU) – Denmark
\item University of West of England (UWE) – Bristol, Great Britain
\item Karlsruhe Institute of Technology (KIT) – Germany
\item University of Glasgow (UGL) – Great Britain
\item Università degli Studi di Trento – Italy
\end{itemize}
Il progetto mira a sviluppare un tipo di evoluzione artificiale e tecnologica da utilizzare per la progettazione di sistemi funzionali composti da tre forme di \emph{living technology}: vita chimica artificiale, microrganismi viventi e da reti di reazioni chimiche complesse usate per migliorare il processamento e la purificazione di acque di scarico per la generazione di energia.
Il progetto EVOBLISS combina gli approcci scientifici della robotica, dell'intelligenza artificiale, della chimica e della microbiologia per sfruttare al meglio le avanguardie di queste discipline. Lo scopo finale è quello di produrre una piattaforma robotica facilmente utilizzabile e personalizzabile per l'evoluzione artificiale di nuovi materiali e per l'ottimizzazione delle performance di sistemi psicochimici e microbici. Il prodotto su cui si è lavorato è EvoBot, un robot in grado di gestire liquidi e di fornire riscontri in tempo reale. 

\subsection{EVOBLISS a Trento}
\label{sec:trento}
Il laboratorio di Biologia Artificiale del CIBIO - Centre for Integrative Biology dell'Università di Trento è parte integrante del progetto EVOBLISS.
Qui ci si occupa di sviluppare diversi tipi di cellule artificiali basandosi su emulsioni \emph{droplet-based}. 
Le \emph{droplets} a cui si fa riferimento sono composte da \emph{1-Decanolo} $(C_{10}H_{21}OH)$. Il decanolo è formato da una catena lineare di alcol grasso con 10 atomi di carbonio.\cite{decanolo} Essendo incolore vi si aggiunge \emph{Oil Red O} \cite{oilredo}, un colorante solubile con i grassi che lo rende visibile all'occhio umano e ci permette di tenerne traccia all'interno del sistema. Le \emph{droplets} analizzate hanno volume pari a $30\mu l$, il volume è stato scelto per avere una migliore  tracciabilità dal programma di analisi di \emph{computer vision}.
Le caratteristiche prese in analisi sono simili a quelle che possono caratterizzare un essere vivente: l'auto-movimento, l'auto-divisione, la trasformazione biochimica e le dinamiche di gruppo.
\\La piattaforma è stata qui adattata alla necessità di monitorare il movimento delle \emph{droplets} a partire dal momento in cui queste escono da un sistema di equilibrio, studiando l'ottimizzazione dei parametri che definiscono l'ambiente in cui queste si muovono. Lo scopo ultimo è quello di incrementare la conoscenza delle \emph{living technologies} e di ideare e sfruttare al meglio sistemi bio-ibridi innovativi.
Ulteriori studi hanno visto l'implementazione di programmi per il riconoscimento di droplets multiple. 
 















